\Introduction

\textbf{Цель работы}: построение гистограммы и эмпирической функции распределения.

\textbf{Содержание работы}

\begin{enumerate}
	\item Для выборки объема $n$ из генеральной совокупности $X$ реализовать в виде программы на ЭВМ:

	\begin{enumerate}[a)]
		\item вычисление максимального значения $M_{max}$ и минимального значения $M_{min}$;
		
		\item размаха $R$ выборки;
		
		\item вычисление оценок $\hat{\mu}$ и $S^2$ математического ожидания $MX$ и дисперсии $DX$;
		
		\item группировку значений выборки в $m=\lfloor\log_2n\rfloor+2$ интервала;
		
		\item построение на одной координатной плоскости гистограммы и графика функции плотности распределения вероятностей нормальной случайной величины с математическим
ожиданием $\hat{\mu}$ и дисперсией $S^2$;
		
		\item построение на другой координатной плоскости графика эмпирической функции распределения и функции распределения нормальной случайной величины с математическим ожиданием $\hat{\mu}$ и дисперсией $S^2$.
	\end{enumerate}

	\item Провести вычисления и построить графики для выборки из индивидуального варианта.
\end{enumerate}

\textbf{Содержание отчета}

\begin{enumerate}
	\item формулы для вычисления величин $M_{max},~M_{min},~R,~\hat{\mu},~S^2$;
	\item определение эмпирической плотности и гистограммы;
	\item определение эмпирической функции распределения;
	\item текст программы;
	\item результаты расчетов для выборки из индивидуального варианта.
\end{enumerate}
