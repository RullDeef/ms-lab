\chapter{Задача}

\textbf{Условие задачи}

Математическое ожидание скорости ветра у земли в данной местности составляет 8 км/ч. Оценить вероятность того, что скорость ветра превысит 20 км/ч и что она
будет меньше 50 км/ч. Как изменятся искомые вероятности, если будет известно, что среднее квадратичное отклонение скорости ветра равно 2 км/ч?

\textbf{Решение}



\chapter{Задача}

\textbf{Условие задачи}

С использованием метода моментов для случайной выборки $\overrightarrow{X} = (X_1,~\dots,~X_n)$ из генеральной совокупности $X$ найти точечные оценки указанных параметров заданного закона распределения:

\begin{equation*}
	f_X(x) = \frac{\sqrt{\theta}}{\Gamma(\alpha)} x^{\alpha - 1}e^{-\lambda x},~~~x>0
\end{equation*}

\textbf{Решение}

\chapter{Задача}

\textbf{Условие задачи}

С использованием метода максимального правдоподобия для случайной выборки $\overrightarrow{X}
= (X_1, ~\dots,~X_n)$ из генеральной совокупности $X$ найти точечные оценки параметров заданного
закона распределения. Вычислить выборочные значения найденных оценок для выборки $\overrightarrow{x}_5 = (x_1,~\dots,~x_5)$.

\textbf{Закон распределения}

\begin{equation*}
	f_X(x)=\frac{1}{\theta\sqrt{2\pi x}}e^{-x/(2\theta^2)},~~~x>0
\end{equation*}

\textbf{Выборка}

\begin{equation*}
	\overrightarrow{x}_5 = (4.2,~7.8,~16.3,~11.6,~8.3)
\end{equation*}

\chapter{Задача}

\textbf{Условие задачи}

Оценка значений сопротивления партии из $n=100$ однотипных резисторов, составила $x_n=10$ кОм. Считая, что ошибки измерений распределены по нормальному закону
со среднеквадратичным отклонением $\sigma=1$ кОм, найти вероятность того, что для резисторов
всей партии среднее значение сопротивления лежит в пределах $10\pm0.1$ кОм.
