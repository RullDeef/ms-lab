\Introduction

\textbf{Цель работы}: построение доверительных интервалов для математического ожидания и дисперсии нормальной случайной величины.

\textbf{Содержание работы}

\begin{enumerate}[1.]
	\item Для выборки объема $n$ из генеральной совокупности $X$ реализовать в виде программы на ЭВМ:

	\begin{enumerate}[a)]
		\item вычисление точечных оценок $\hat{\mu}(\overrightarrow{x_n})$ и $S^2(\overrightarrow{x_n})$ математического ожидания $MX$ и дисперсии $DX$ соответственно;
		
		\item вычисление нижней и верхней границ $\underline{\mu}(\overrightarrow{x_n})$, $\overline{\mu}(\overrightarrow{x_n})$ для $\gamma$-доверительного интервала для математического ожидания $MX$;
		
		\item вычисление нижней и верхней границ $\underline{\sigma}^2(\overrightarrow{x_n})$, $\overline{\sigma}^2(\overrightarrow{x_n})$ для $\gamma$-доверительного интервала для дисперсии $DX$;
	\end{enumerate}

	\item вычислить $\hat{\mu}$ и $S^2$ для выборки из индивидуального варианта.
	
	\item для заданного пользователем уровня доверия $\gamma$ и $N$ -- объема выборки из индивидуального варианта:
	
	\begin{enumerate}[a)]
		\item на координатной плоскости $Oyn$ построить прямую $y = \hat{\mu}(\overrightarrow{x_N})$, также графики функций $y = \hat{\mu}(\overrightarrow{x_n})$, $y = \underline{\mu}(\overrightarrow{x_n})$, и $y = \overline{\mu}(\overrightarrow{x_n})$ как функций объема $n$ выборки, где $n$ изменяется от $1$ до $N$;
		
		\item на другой координатной плоскости $Ozn$ построить прямую $z = S^2(\overrightarrow{x_N})$, также графики функций $z = S^2(\overrightarrow{x_n})$, $z = \underline{\sigma}^2(\overrightarrow{x_n})$ и $z = \overline{\sigma}^2(\overrightarrow{x_n})$ как функций объема $n$ выборки, где $n$ изменяется от $1$ до $N$. 
	\end{enumerate}
\end{enumerate}

\textbf{Содержание отчета}

\begin{enumerate}[1.]
	\item определение $\gamma$-доверительного интервала для значения параметра распределения случайной величины;
	\item формулы для вычисления границ $\gamma$-доверительного интервала для математического ожидания и дисперсии нормальной случайной величины;
	\item текст программы;
	\item результаты расчетов и графики для выборки из индивидуального варианта (при построении графиков принять $\gamma = 0.9$).
\end{enumerate}
